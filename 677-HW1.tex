% Options for packages loaded elsewhere
\PassOptionsToPackage{unicode}{hyperref}
\PassOptionsToPackage{hyphens}{url}
%
\documentclass[
]{article}
\usepackage{lmodern}
\usepackage{amssymb,amsmath}
\usepackage{ifxetex,ifluatex}
\ifnum 0\ifxetex 1\fi\ifluatex 1\fi=0 % if pdftex
  \usepackage[T1]{fontenc}
  \usepackage[utf8]{inputenc}
  \usepackage{textcomp} % provide euro and other symbols
\else % if luatex or xetex
  \usepackage{unicode-math}
  \defaultfontfeatures{Scale=MatchLowercase}
  \defaultfontfeatures[\rmfamily]{Ligatures=TeX,Scale=1}
\fi
% Use upquote if available, for straight quotes in verbatim environments
\IfFileExists{upquote.sty}{\usepackage{upquote}}{}
\IfFileExists{microtype.sty}{% use microtype if available
  \usepackage[]{microtype}
  \UseMicrotypeSet[protrusion]{basicmath} % disable protrusion for tt fonts
}{}
\makeatletter
\@ifundefined{KOMAClassName}{% if non-KOMA class
  \IfFileExists{parskip.sty}{%
    \usepackage{parskip}
  }{% else
    \setlength{\parindent}{0pt}
    \setlength{\parskip}{6pt plus 2pt minus 1pt}}
}{% if KOMA class
  \KOMAoptions{parskip=half}}
\makeatother
\usepackage{xcolor}
\IfFileExists{xurl.sty}{\usepackage{xurl}}{} % add URL line breaks if available
\IfFileExists{bookmark.sty}{\usepackage{bookmark}}{\usepackage{hyperref}}
\hypersetup{
  pdftitle={677 HW1},
  pdfauthor={Chi Zhang},
  hidelinks,
  pdfcreator={LaTeX via pandoc}}
\urlstyle{same} % disable monospaced font for URLs
\usepackage[margin=1in]{geometry}
\usepackage{color}
\usepackage{fancyvrb}
\newcommand{\VerbBar}{|}
\newcommand{\VERB}{\Verb[commandchars=\\\{\}]}
\DefineVerbatimEnvironment{Highlighting}{Verbatim}{commandchars=\\\{\}}
% Add ',fontsize=\small' for more characters per line
\usepackage{framed}
\definecolor{shadecolor}{RGB}{248,248,248}
\newenvironment{Shaded}{\begin{snugshade}}{\end{snugshade}}
\newcommand{\AlertTok}[1]{\textcolor[rgb]{0.94,0.16,0.16}{#1}}
\newcommand{\AnnotationTok}[1]{\textcolor[rgb]{0.56,0.35,0.01}{\textbf{\textit{#1}}}}
\newcommand{\AttributeTok}[1]{\textcolor[rgb]{0.77,0.63,0.00}{#1}}
\newcommand{\BaseNTok}[1]{\textcolor[rgb]{0.00,0.00,0.81}{#1}}
\newcommand{\BuiltInTok}[1]{#1}
\newcommand{\CharTok}[1]{\textcolor[rgb]{0.31,0.60,0.02}{#1}}
\newcommand{\CommentTok}[1]{\textcolor[rgb]{0.56,0.35,0.01}{\textit{#1}}}
\newcommand{\CommentVarTok}[1]{\textcolor[rgb]{0.56,0.35,0.01}{\textbf{\textit{#1}}}}
\newcommand{\ConstantTok}[1]{\textcolor[rgb]{0.00,0.00,0.00}{#1}}
\newcommand{\ControlFlowTok}[1]{\textcolor[rgb]{0.13,0.29,0.53}{\textbf{#1}}}
\newcommand{\DataTypeTok}[1]{\textcolor[rgb]{0.13,0.29,0.53}{#1}}
\newcommand{\DecValTok}[1]{\textcolor[rgb]{0.00,0.00,0.81}{#1}}
\newcommand{\DocumentationTok}[1]{\textcolor[rgb]{0.56,0.35,0.01}{\textbf{\textit{#1}}}}
\newcommand{\ErrorTok}[1]{\textcolor[rgb]{0.64,0.00,0.00}{\textbf{#1}}}
\newcommand{\ExtensionTok}[1]{#1}
\newcommand{\FloatTok}[1]{\textcolor[rgb]{0.00,0.00,0.81}{#1}}
\newcommand{\FunctionTok}[1]{\textcolor[rgb]{0.00,0.00,0.00}{#1}}
\newcommand{\ImportTok}[1]{#1}
\newcommand{\InformationTok}[1]{\textcolor[rgb]{0.56,0.35,0.01}{\textbf{\textit{#1}}}}
\newcommand{\KeywordTok}[1]{\textcolor[rgb]{0.13,0.29,0.53}{\textbf{#1}}}
\newcommand{\NormalTok}[1]{#1}
\newcommand{\OperatorTok}[1]{\textcolor[rgb]{0.81,0.36,0.00}{\textbf{#1}}}
\newcommand{\OtherTok}[1]{\textcolor[rgb]{0.56,0.35,0.01}{#1}}
\newcommand{\PreprocessorTok}[1]{\textcolor[rgb]{0.56,0.35,0.01}{\textit{#1}}}
\newcommand{\RegionMarkerTok}[1]{#1}
\newcommand{\SpecialCharTok}[1]{\textcolor[rgb]{0.00,0.00,0.00}{#1}}
\newcommand{\SpecialStringTok}[1]{\textcolor[rgb]{0.31,0.60,0.02}{#1}}
\newcommand{\StringTok}[1]{\textcolor[rgb]{0.31,0.60,0.02}{#1}}
\newcommand{\VariableTok}[1]{\textcolor[rgb]{0.00,0.00,0.00}{#1}}
\newcommand{\VerbatimStringTok}[1]{\textcolor[rgb]{0.31,0.60,0.02}{#1}}
\newcommand{\WarningTok}[1]{\textcolor[rgb]{0.56,0.35,0.01}{\textbf{\textit{#1}}}}
\usepackage{graphicx,grffile}
\makeatletter
\def\maxwidth{\ifdim\Gin@nat@width>\linewidth\linewidth\else\Gin@nat@width\fi}
\def\maxheight{\ifdim\Gin@nat@height>\textheight\textheight\else\Gin@nat@height\fi}
\makeatother
% Scale images if necessary, so that they will not overflow the page
% margins by default, and it is still possible to overwrite the defaults
% using explicit options in \includegraphics[width, height, ...]{}
\setkeys{Gin}{width=\maxwidth,height=\maxheight,keepaspectratio}
% Set default figure placement to htbp
\makeatletter
\def\fps@figure{htbp}
\makeatother
\setlength{\emergencystretch}{3em} % prevent overfull lines
\providecommand{\tightlist}{%
  \setlength{\itemsep}{0pt}\setlength{\parskip}{0pt}}
\setcounter{secnumdepth}{-\maxdimen} % remove section numbering

\title{677 HW1}
\author{Chi Zhang}
\date{}

\begin{document}
\maketitle

To show how did the author of the book get critical value m that while
m=69 to thwarts type 1 error and m= 73 to thwarts type 2 error, I write
a function named Plot to show the process of finding these two critical
value. I set p as a sequence from 0.4 to 1 by 0.01 each. I used
probability density function to get desired a(p), the sum probability we
reject the null hypothesis is true.

\begin{Shaded}
\begin{Highlighting}[]
\KeywordTok{library}\NormalTok{(ggplot2)}
\NormalTok{Plot <-}\ControlFlowTok{function}\NormalTok{(m)\{}
\NormalTok{n <-}\DecValTok{100}
\NormalTok{p <-}\KeywordTok{seq}\NormalTok{(}\FloatTok{0.4}\NormalTok{,}\DecValTok{1}\NormalTok{,}\FloatTok{0.01}\NormalTok{)}
\NormalTok{ap <-}\KeywordTok{c}\NormalTok{()}
\ControlFlowTok{for}\NormalTok{ (i }\ControlFlowTok{in}\NormalTok{ p)\{}
\NormalTok{  a <-}\StringTok{ }\KeywordTok{sum}\NormalTok{(}\KeywordTok{dbinom}\NormalTok{(m}\OperatorTok{:}\NormalTok{n,n,i))}
\NormalTok{  ap <-}\KeywordTok{c}\NormalTok{(ap,a)}
\NormalTok{\}}
\KeywordTok{ggplot}\NormalTok{()}\OperatorTok{+}\KeywordTok{geom_line}\NormalTok{(}\KeywordTok{aes}\NormalTok{(p,ap))}\OperatorTok{+}\KeywordTok{geom_rect}\NormalTok{(}\KeywordTok{aes}\NormalTok{(}\DataTypeTok{xmin =} \FloatTok{0.6}\NormalTok{, }\DataTypeTok{xmax =} \FloatTok{0.8}\NormalTok{, }\DataTypeTok{ymin =} \FloatTok{0.05}\NormalTok{, }\DataTypeTok{ymax =} \FloatTok{0.95}\NormalTok{),}\DataTypeTok{alpha=}\DecValTok{0}\NormalTok{,}\DataTypeTok{color=}\StringTok{"grey"}\NormalTok{)}\OperatorTok{+}\KeywordTok{ggtitle}\NormalTok{(}\StringTok{"The power curve"}\NormalTok{)}\OperatorTok{+}\KeywordTok{ylab}\NormalTok{(}\StringTok{"alpha"}\NormalTok{)}
\NormalTok{\}}
\end{Highlighting}
\end{Shaded}

I started to plug value of m from 0-100 to find the critical values
which pass the bottom left and top right of the rectangle in Figure 3.7.
As the m increases, the line moves to the right. Examples of m= 65, 70,
75.

\begin{Shaded}
\begin{Highlighting}[]
\KeywordTok{Plot}\NormalTok{(}\DecValTok{65}\NormalTok{)}
\end{Highlighting}
\end{Shaded}

\includegraphics{677-HW1_files/figure-latex/unnamed-chunk-2-1.pdf}

\begin{Shaded}
\begin{Highlighting}[]
\KeywordTok{Plot}\NormalTok{(}\DecValTok{70}\NormalTok{)}
\end{Highlighting}
\end{Shaded}

\includegraphics{677-HW1_files/figure-latex/unnamed-chunk-2-2.pdf}

\begin{Shaded}
\begin{Highlighting}[]
\KeywordTok{Plot}\NormalTok{(}\DecValTok{75}\NormalTok{)}
\end{Highlighting}
\end{Shaded}

\includegraphics{677-HW1_files/figure-latex/unnamed-chunk-2-3.pdf}

After narrowing down numbers, I found while m=69, the line just enters
the bottom, and whole m=73, the line just leaves from the top.
Therefore, 69 is the smallest m thwarts type 1 error and 73 is the
largest which thwarts a type 2 error.

\begin{Shaded}
\begin{Highlighting}[]
\NormalTok{n <-}\DecValTok{100}
\NormalTok{p <-}\KeywordTok{seq}\NormalTok{(}\FloatTok{0.4}\NormalTok{,}\DecValTok{1}\NormalTok{,}\FloatTok{0.01}\NormalTok{)}
\NormalTok{ap69 <-}\KeywordTok{c}\NormalTok{()}
\ControlFlowTok{for}\NormalTok{ (i }\ControlFlowTok{in}\NormalTok{ p)\{}
\NormalTok{  a <-}\StringTok{ }\KeywordTok{sum}\NormalTok{(}\KeywordTok{dbinom}\NormalTok{(}\DecValTok{69}\OperatorTok{:}\NormalTok{n,n,i))}
\NormalTok{  ap69 <-}\KeywordTok{c}\NormalTok{(ap69,a)}
\NormalTok{\}}
\NormalTok{ap73 <-}\KeywordTok{c}\NormalTok{()}
\ControlFlowTok{for}\NormalTok{ (i }\ControlFlowTok{in}\NormalTok{ p)\{}
\NormalTok{  a <-}\StringTok{ }\KeywordTok{sum}\NormalTok{(}\KeywordTok{dbinom}\NormalTok{(}\DecValTok{73}\OperatorTok{:}\NormalTok{n,n,i))}
\NormalTok{  ap73 <-}\KeywordTok{c}\NormalTok{(ap73,a)}
\NormalTok{\}}
\KeywordTok{ggplot}\NormalTok{()}\OperatorTok{+}\KeywordTok{geom_line}\NormalTok{(}\KeywordTok{aes}\NormalTok{(p,ap69))}\OperatorTok{+}\KeywordTok{geom_line}\NormalTok{(}\KeywordTok{aes}\NormalTok{(p,ap73))}\OperatorTok{+}\KeywordTok{geom_rect}\NormalTok{(}\KeywordTok{aes}\NormalTok{(}\DataTypeTok{xmin =} \FloatTok{0.6}\NormalTok{, }\DataTypeTok{xmax =} \FloatTok{0.8}\NormalTok{, }\DataTypeTok{ymin =} \FloatTok{0.05}\NormalTok{, }\DataTypeTok{ymax =} \FloatTok{0.95}\NormalTok{),}\DataTypeTok{alpha=}\DecValTok{0}\NormalTok{,}\DataTypeTok{color=}\StringTok{"grey"}\NormalTok{)}\OperatorTok{+}\KeywordTok{ggtitle}\NormalTok{(}\StringTok{"The power curve"}\NormalTok{)}\OperatorTok{+}\KeywordTok{ylab}\NormalTok{(}\StringTok{"alpha"}\NormalTok{)}
\end{Highlighting}
\end{Shaded}

\includegraphics{677-HW1_files/figure-latex/unnamed-chunk-3-1.pdf}

\end{document}
